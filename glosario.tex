\secnumberlesssection{GLOSARIO}

Aquí se deben colocar las siglas mencionadas en el trabajo y su explicación, por orden alfabético. Por ejemplo: \\

{\setlength{\parskip}{0cm} % Para evitar saltar entre cada elemento nombrado.
%Colocar aquí siglas:
\textbf{A priori}: previamente, con anterioridad.

\textbf{Accuracy}: para un modelo predictivo, se entiende como la razón entre el número de predicciones correctas y el número total de predicciones hechas.

\textbf{Cluster}: conjunto de elementos similares entre sí.

\textbf{Clustering}: la tarea encontrar clusters, dado un conjunto de elementos.

\textbf{Corpus}: en análisis de textos se entiende como el conjunto de documentos a estudiar.

\textbf{Dataset}: en minería de datos se entiende como el conjunto de datos a estudiar. 

\textbf{Features}: en minería de datos, se entiende como las características de cada registro a estudiar. Por ejemplo estatura y edad de una persona.

\textbf{Framework}: es una estructura real o conceptual destinada a servir como soporte o guía para la construcción de algo.

\textbf{Json}: es un formato de intercambio de datos ligero muy utilizado en el desarrollo web.

\textbf{Keywords}: palabras más relevantes dentro de un texto.

\textbf{Missing Values}: en minería de datos, son los features ,que por alguna razón, no están almacenados en ciertos registros.

\textbf{Stakeholders}: personas o entidades interesadas en el desarrollo de un proyecto en particular.

\textbf{Stopwords}: son las palabras más comunes en un idioma en particular. Por ejemplo artículos y preposiciones en español.

\textbf{Tokens}: es una palabra que tiene significado por si sola.

\textbf{Trade-off}: un equilibrio logrado entre dos características deseables pero incompatibles; un compromiso.
}
