\secnumbersection{DEFINICIÓN DEL PROBLEMA}
\subsection{Descripción de la organización}
Hatch es una organización multinacional de origen canadiense con presencia en los cinco continentes, cuya principal actividad es realizar proyectos de ingeniería en mercados tales como: Metales, energía, infraestructura, digital y de inversión en más de 150 países alrededor de todo el mundo.

Uno de los principales desafíos que se ha auto-impuesto la compañía es el de la digitalización con la frase “Desde la digitalización a la transformación”. Para lograr esto Hatch ha trabajado estos últimos años en proyectos que ayudan a sus clientes a digitalizar procesos cotidianos, con el fin de alcanzar 4 grandes objetivos:

\begin{itemize}
    \item Reducir los gastos de capital y de operaciones
    \item Mejorar la productividad
    \item Optimizar las operaciones
    \item Aumentar la seguridad
\end{itemize}

Un proyecto destacado de Hatch, que va en la línea de lo que se comentó anteriormente, es “Muestreo hídricos en profundidad con Drones” para Goldcorp Canada Ltd., Kinross, National Oilwell Varco y la División de Protección Ambiental de Nevada en Canadá y Estados Unidos. En él se buscaba automatizar el proceso de muestreo hídrico que debe realizar una minera en todos los lagos y los cuerpos de agua con la que sus operaciones entran en contacto. 

Para realizar lo anterior, grupos de personas entraban en botes a tomar muestras frecuentemente con el fin de monitorear la evolución de la calidad del agua. Sin embargo esto traía muchos problemas de seguridad para el equipo humano que debía adentrarse en la aventura, tales como: ahogamiento, hipotermia, contacto directo de la piel con sustancias peligrosas, entre otros. 

La solución propuesta por Hatch ante eventual problemática fue utilizar un sistema de muestreo hídrico aéreo controlado de forma remota (drones). De esta forma se pudo mejorar la productividad, aumentar la seguridad de los trabajadores de la minera, e incluso optimizar la operación ya que el sistema automatizado resultó ser más eficiente.

\subsection{Descripción de la situación actual}
Como se mencionó en la sección anterior una de las premisas con la que actualmente está operando la empresa es “Desde la digitalización a la transformación”. Es por esto que, además de ayudar a clientes externos de la organización a mejorar sus procesos, la compañía desea mejorar sus procesos internos utilizando herramientas digitales. Hatch Chile desea explorar la posibilidad de extraer nueva información a partir de datos históricos almacenados por diversos sistemas de información.

 Uno de los sistemas de información que más datos almacena dentro de la compañía es el de gestión documental, donde se encuentran almacenadas las propuestas técnicas de trabajo. Una propuesta en Hatch Chile es una respuesta al requerimiento de un cliente llamado licitación; este documento detalla cada uno de los aspectos técnicos y económicos con los que Hatch resolvería la problemática planteada. En resumen es lo que ofrece Hatch a un posible cliente para que esté le asigne el trabajo y, a la vez, los recursos para la ejecución de un proyecto. 
 
 En una propuesta trabajan profesionales de todas las áreas relativas al proyecto en cuestión. Cada área genera una parte de la propuesta y luego un grupo encargado se preocupa de \emph{compilar} todos los elementos y generar la propuesta final.

Al día de hoy la empresa en estudio tiene almacenadas propuestas técnicas desde el año 2008, alcanzando la no despreciable cifra de aproximadamente diez mil documentos. Si bien, se extrae información realizando análisis de manera manual sobre el desarrollo y resultado final de las propuestas, esto resulta ser un proceso tedioso y con un beneficio muy limitado.

Lo anterior ocurre debido a que en la compañía, no existen expertos en análisis de datos quienes podrían someter estos activos a procesos de minería de textos e incluso explotar los documentos con herramientas de análisis de textos. En consecuencia, las personas que trabajan en el equipo de desarrollo de propuestas técnicas, no conocen las nuevas tecnologías y formas que existen de analizar múltiples documentos de manera más rápida y eficiente, por lo que claramente continúan ejecutando sus procesos de forma tradicional.

\subsection{Problema}
Considerando lo enunciado en la sección anterior, se puede resumir la problemática de la empresa en la siguiente frase: ``Hatch no utiliza técnicas de análisis de textos durante el proceso de generación de propuestas técnicas''. Esto provoca que el análisis de propuestas ya generadas y propuestas en estado de producción, sea más lento a la hora de analizar grandes volúmenes de documentos y por ende, requiera un mayor esfuerzo por parte del equipo a la hora de realizar dichos análisis. Por otro lado, el uso de este tipo de técnicas, puede encontrar información que haya sido pasada por alto a la hora de realizar los mismos análisis utilizando métodos tradicionales. Es por esto que se hace evidente un estudio, sobre la factibilidad de utilizar dichas técnicas en documentos técnicos. Esto para ver alcances, casos de uso e incluso la extracción de información que ayude a generar directrices en el diseño de nuevas propuestas.


\subsection{Objetivos}

\subsubsection{Objetivo general} Analizar y validar el uso de técnicas de análisis de textos para generar directrices en el diseño de propuestas de proyectos ingenieriles de una empresa del rubro, que ayuden a su futura aprobación.


\subsubsection{Objetivos específicos}
\begin{itemize}
    \item Investigar y evaluar algoritmos y técnicas de análisis de textos, para aplicarlos al conjunto de datos de la empresa.
    \item Aplicar las técnicas y algoritmos seleccionados a un conjunto de propuestas técnicas, con el fin de extraer patrones que permitan distinguir entre propuestas ganadoras y perdedoras.
    \item Construir un documento de buenas prácticas, para la generación de propuestas para ser utilizado como guía a la hora de escribir un nuevo documento con el fin de  agilizar el proceso de producción de una nueva propuesta.
    \item  Validar la propuesta entregada utilizando los conocimientos de los expertos en el dominio, para verificar la correctitud de la solución.
    
\end{itemize}
